\documentclass[conference]{IEEEtran}

% correct bad hyphenation here
\hyphenation{op-tical net-works semi-conduc-tor}
\usepackage{amsmath}
\usepackage{comment}
\usepackage{graphicx}
%\usepackage{cases}
%\usepackage{subeqnarray}

\usepackage{color}
\usepackage{xcolor}
\newcommand{\todo}[1]{\textcolor{red}{[TODO: #1]}}

\begin{document}
%
% paper title
% Titles are generally capitalized except for words such as a, an, and, as,
% at, but, by, for, in, nor, of, on, or, the, to and up, which are usually
% not capitalized unless they are the first or last word of the title.
% Linebreaks \\ can be used within to get better formatting as desired.
% Do not put math or special symbols in the title.
\title{Multiple Object Tracking with DeepSORT}
%
%
% author names and IEEE memberships
% note positions of commas and nonbreaking spaces ( ~ ) LaTeX will not break
% a structure at a ~ so this keeps an author's name from being broken across
% two lines.
% use \thanks{} to gain access to the first footnote area
% a separate \thanks must be used for each paragraph as LaTeX2e's \thanks
% was not built to handle multiple paragraphs
%

\author{
    Zhengke Wu,
    Jiabin Fang,
    Tianxiao Shen
}

% The paper headers
% \markboth{Journal of \LaTeX\ Class Files,~Vol.~13, No.~9, September~2014}%
% {Shell \MakeLowercase{\textit{et al.}}: Bare Demo of IEEEtran.cls for Journals}
% The only time the second header will appear is for the odd numbered pages
% after the title page when using the twoside option.
%
% *** Note that you probably will NOT want to include the author's ***
% *** name in the headers of peer review papers.                   ***
% You can use \ifCLASSOPTIONpeerreview for conditional compilation here if
% you desire.


% make the title area
\maketitle

% As a general rule, do not put math, special symbols or citations
% in the abstract or keywords.
\begin{abstract}
    This is the report of the course project of EI339 Artificial Intelligence, concerning the problem of multiple object tracking. Simple Online and Realtime Tracking (SORT) is a pragmatic
    approach to multiple object tracking with a focus on simple,
    effective algorithms. Deep SORT integrates appearance information to improve the performance of SORT. \\
    In this paper, we studied and re-implemented the Deep SORT method, and proposed some improvements to it.
\end{abstract}

% Note that keywords are not normally used for peerreview papers.
% \begin{IEEEkeywords}
% IEEEtran, journal, \LaTeX, paper, template.
% \end{IEEEkeywords}


% \IEEEpeerreviewmaketitle



\section{Introduction}

This is the report of the course project of EI339 Artificial Intelligence. We should admit that it is not one of publishable quality, although we paid a lot, e.g., writing the report in publishable format. As for the three evaluation levels, we think our work can be classified into Good, for \todo{brief supporting points}.

Multiple Object Tracking (MOT) plays an important role in solving many fundamental problems in video analysis and computer vision. Its main task is to find and identify moving objects in a sequence of images. Depending on how objects are initialized, the strategies of MOT can be classified into two sets: Detection-Based Tracking (DBT) and Detection-Free Tracking (DFT), of which the first one is the leading paradigm in this field of research. The DBT methods employ two steps: Objection Detection and Data Association, i.e., detecting objects of interest in each frame of a video first and then obtrain the tracks of the detected objects across frames according to their correspondence. In this way, the MOT problem can be viewed as a data association problem.

MOT can also be categorized into online tracking and offline tracking. The difference is whether or not observations from future frames are utilized when handling the current frame. Online algorithms generally performs worse than offline algorithms, which is natural, but is required in real time scenarios. Our focus is on online tracking during this project.

Deep SORT \cite{Wojke2017simple} is an extension to Simple Real time Tracker (SORT) \cite{Bewley2016_sort}, and is one of the most popular and the most widely used object tracking frameworks. It achieved competitive performance with simple and elegent ideas, and can act as a baseline in the field of MOT. In this project, we are asked to study and re-implement the classical Deep SORT algorithm and achieve some improvements based on the Deep SORT framework.

\section{Studying Deep Sort}

\subsection{The ideas of Deep SORT}

Simple online and realtime tracking (SORT) is a very simple framework that performs Kalman filtering in image space and frame-by-frame data association using the Hungarian method with an association metric that measures bounding box overlap. This simple approach achieved favorable performance at high frame rates. It used state of the art convolutional neural network (CNN) based object detector \cite{ren2015faster}.

To be more specific, Hungarian algorithm \cite{kuhn1955hungarian} is a very classic algorithm to find a maximum cardinality matching of minimum cost in the bipartite graph matching problem. And Kalman filter \cite{kalman1960new} is an algorithm that uses a series of measurements observed over time to predict the motion of an object. SORT approximates the inter-frame displacements of each object with a linear constant velocity model which is independent of other objects and camera motion. The state of each target is modelled as:

\[
    \boldsymbol{x} = [u,v,s,r,\dot{u},\dot{v},\dot{s}]^T, 
\]

\noindent where $u$ and $v$ represent the horizontal and vertical pixel location of the centre of the target, while the scale $s$ and $r$ represent the scale (area) and the aspect ratio of the target’s bounding box respectively.

Now the idea behind SORT seems rather simple, but it did make some contributions to the development of MOT at that time. Its code was also made open source, and provided a new baseline for the field of MOT.

Deep SORT took another step forward. The main contribution of Deep SORT beyond the original framework of SORT is a deep association metric learned on a largescale person re-identification dataset. Besides, it integrates appearance information to recover identities after long-term occlusions, when motion is less discriminative. It extensions reduce the number of identity switches by 45\%, which improves the defects of SORT, achieving overall competitive performance at high frame rates.

We have to stress that Detaction-Based Tracking algorithms like SORT and Deep SORT rely highly on the results of object detection, and appealing to more powerful detector or one trained in a more targeted way can easily provide better tracking results, as the authors of SORT admitted.

\subsection{Implementation}

% \begin{figure}[t]
%     \centering
%     \includegraphics[width=\linewidth]{fig/1.pdf}
%     \caption{A picture.}
%     \label{fig:example}
% \end{figure}

% see Fig.~\ref{fig:example}. 

\section{Our Improvements}

\section{Conclusion}

In this project, we dove into the field of Multiple Object Tracking (MOT). It is both interesting and challenging. Our work is based on the credible Deep SORT method. We presented ...

\bibliographystyle{IEEEtran}
\bibliography{Ref}

\end{document}